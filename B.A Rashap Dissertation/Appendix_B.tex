\chapter{Surface Evolution Simulation}\appcaption{Appendix B}

\tab The evolution of the surface during an etch was simulated by using a string model. The
following is the C implementation string model used in this research. The header file is
presented first followed by the main program.


\subsection{string.h}

\begin{lstlisting}[language=C]
	/* Header File for String Model */
	
	/* Globally Defined Constants */
	
	#define PI		3.14159
	
	
	/* Define Global Variables */
	
	int		NODES,ITERATIONS,NNODE;
	float	WIDTH,TIME,ISO,ANISO,DMIN,DMAX;
	float	PR_HEIGHT,PR_WIDTH,PR_ISO,PR_ANISO
	
	/* Difine Structures */
	
	strict position
	{
		float x,y;
	};

	struct node
	{
		struct node *prev;
		struct position last;
		float theta;
		in nodenum;
		struct position new;
		struct node *next;
	};

	/* Declare Functions */
	
	void initialize(struct node *pntr);
	void angle(struct node *pntr);
	void iso_evolve(struct node *pntr);
	void ansio_evolve(struct node *pntr);
	void evolve(struct node *pntr);
	void update(struct node *pntr);
	void addnode(struct node *pntr);
	void remnode(struct node *pntr);
	float dist(struct node *pntr);
	int leftbound(struct node *pntr);
	int rightbound(struct node *pntr);
	void PRinitialize(struct node *pntr);
	
\end{lstlisting}

\newpage

\subsection{string.c}

\begin{lstlisting}[language=C]
	/* String Model Programs		*/
	/*								*/
	/* Created By: Brian Rashap		*/
	/* Date		   6/20/94			*/
	
	#include <stdio.h>
	#include <math.h>
	#include "string.h"
	
	main()
	{
	 struct node first,*point;
	 int i,k,numofnodes,*nnum;
	 float distance,timestep;
		
	 char *calloc ();
		
	/*	Scan in Simulation Data 				*/
	/*	WIDTH = mastk linewidth (um)			*/
	/*	NODES = initial number of nodes			*/
	/*	ITERATIONS = iterations of evolution	*/
	/*	TIME = etch length (seconds)			*/
	/*	ISO = isotropic etch rate (A/sec)		*/
	/*	ANISO = anisotropic etch rate (A/sec)	*/
	/*	DMAX = max. distance between nodes		*/
	/*	DMIN = min. distance between nodes		*/
		
	  scanf("%f %d %d",&WIDTH, &NODES, &ITERATIONS);
	  scanf("%f %f %f",&TIME,&ISO,&ANISO);
	  scanf("%f %f,&DMAX, &DMIN);
			
	  timestep = TIME/ITERATIONS;	/* Time step in seconds	*/
	  ISO = ISO * timestep;		/* Isotropic ER in A/step	*/
	  ANISO = ANISO * timestep; /* Anisotropic ER in A/step */
	
	/*	Initialize First Node	*/
	
	  point = (struct node *) malloc (sizeof (structnode));
	  first.theta = -PI/2;
	  first.next = point;
	  first.last.x = -(WIDTH/2.0);
	  first.last.y = 0.0;
	  first.prev = &first;
	  first.nodenum = 1;
	  
	/*	Initialize Rest of Starting String	*/
	  
	  initialize(first.next);
	  (first.next)->prev = &first;
	  angle(first.next);
	  
	  NNODE = NODES+1;
	  
	/*	Surface Evolution	*/
	
	  for (k = 1; k < ITERATIONS; k++)
	  	{
	/*	print("Calculating Iteration %d \n",k);	*/
		  point = &first;
	  	  evolve(point);
	  	  update(point);
  	  	}
    	
    /*	Display final surface	*/
    
      whole ( point != (struct node *) 0 )
      	{
      	  distance = dist(point);
      	  printf("%d %6.4f %6.4f %6.4f %4.3f %d %d \n",(*point).nodenum, (*point).last.x, (*point).last.y, (*point).theta,distance,leftbound(point),rightbound(point));
      	  point = (*point).next;
      	}
	 }
 
 	/*	Subroutine - initialize string */
 	
 	void initialize(struct node *pntr)
 	{
 	  int i;
 	  
 	  for (i = 2; i < NODES; i++)
 	  	{
 	  	  (*pntr).last.x = -(WIDTH/2.0)+(i-1)*(WIDTH/(NODES-1));
 	  	  (*pntr).last.y = 0.0;
 	  	  (*pntr).theta = 0.0;
 	  	  (*pntr).nodenum = i;
 	  	  (*pntr).next = (struct node *) mallox (size of (strict node));
 	  	  (*pntr->next).prev = pntr;
 	  	  pntr = (*pntr).next;
   		}
   	  (*pntr).theta = i;
   	  (*pntr).nodenum = i;
   	  (*pntr).last.x = (WIDTH/2.0);
   	  (*pntr).last.y = 0.0;
   	  
   	  (*pntr).next = (struct node *) 0;
    }

	/*	Subroutine - Determine surface nurmals	*/
	
	void angle(struct node *pntr)
	{
	  double x1,y1,x2,y2,x3,y3
	  float a12,a23;
	  
	  while ( (*pntr).next != (struct node *) 0 )
	  	{
	  	  x1 = (*pntr->prev).last.x;
	  	  y1 = (*pntr->prev).last.y;
	  	  x2 = (*pntr).last.x;
	  	  y2 = (*pntr).last.y;
	  	  x3 = (*pntr->next).last.x;
	  	  y3 = (*pntr->next).last.y;
	  	  a12 = atan2((y2-y1),(x2-x1));
	  	  a23 = atan2((y3-y2),(x3-x2));
	  	  (*pntr).theta = (a12+a23)/2;
	  	  pntr - (*pntr).next;
  		}
  	  (*pntr).theta = PI/2;
	}

	/*	Isotropic surface evolution	*/
	
	void iso_evolve(struct node *pntr)
	{
	  float th;
	  
	  th = (*pntr).theta;
	  (*pntr).new.x = (*pntr).last.x + ISO*sin(th);
	  (*pntr).new.y = (*pntr).last.y - ISO*cos(th);
	}

	/*	Anisotropix surface evolution	*/
	
	void aniso_evolve(struct node *pntr)
	{
	  (*pntr).new.y = (*pntr).new.y - ANISO;
  	}
  	
  	/*	Evolve Surface on iteration	*/
  	
  	void evolve(struct node *pntr)
  	{
  	  float th;
  	  
  	  while ( pntr != (struct node *) 0 )
  	  	{
  	  	  iso_evolve(pntr);
  	  	  if (leftbound(pntr) == 0 && rightbound(pntr) == 0)
  	aniso_evolve(pntr);
  		  if (dist(pntr) < DMAX)
  	addnode(pntr);
  		  if (dist(pntr) < DMIN)
  			remnode(pntr);
  		  pntr = (*pntr).next;
    	}
  	}
  
  	/*	Update nodes to reflext current positions	*/
  	
  	void update(struct node *pntr)
  	{
  	  struct node *first;
  	  
  	  first = pntr;
  	  while (pntr != (struct node *) 0 )
  	  	{
  	  	  (*pntr).last.x = (*pntr).new.x;
  	  	  (*pntr).last.y = (*pntr).new.y;
  	  	  pntr = (*pntr).next;
    	}
      angle((*first).next);
  	}
  	
  	/*	Determine distance between adjacent nodes	*/
  	
  	float dist(struct node *pntr)
  	{
  	  float distance;
  	  float x1,y1,x2,y2
  	  
  	  x1 = (*pntr->prev).new.x;
  	  y1 = (*pntr->prev).new.y;
  	  x2 = (*pntr).new.x;
  	  y2 = (*pntr).new.y;
  	  
  	  distance = sqrt((y2-y1,2) + pow(x2-x1,2));
  	  return(distance);
  	}
  
  	/*	Add a node to surface string	*/
  	
  	void addnode(struct node *pntr)
  	{
  	  struct node *newnode;
  	  float x1,y1,x2,y2,th;
  	  
  	  newnode = (struct node *) mallox (sizeof (struct node));
  	  
  	  x1 = (*pntr->prev).last.x;
  	  y2 = (*pntr->prev).last.y;
  	  x2 = (*pntr).last.x;
  	  y2 = (*pntr).last.y;
  	  
  	  (*newnode).last.x = (x1+x2)/2;
  	  (*newnode).last.y = (y1+y2)/2;
  	  (*newnode).theta = atan2((y2-y1),(x2-x1));
  	  
  	  (*newnode).prev = (*pntr).prev;
  	  (*newnode).next = pntr;
  	  (*newnode).nodenum = NNODE;
  	  (*pntr->prev).next = newnode;
  	  (*pntr).prev = newnode;
  	  
  	  (NNODE)++;
  	  
  	  iso_evolve(newnode);
  	  if (leftbound(newnode) == 0 && rightbound(newnode) == 0)
  	  	aniso_evolve(newnode);
  	}
  
  	/*	Remove a node from surface string	*/
  	
  	void remnode(struct node *pntr)
  	{
  	  (*pntr->next).prev = (*pntr).prev;
  	  (*pntr->prev).next = (*pntr).next;
  	}
  
  	/*	Determind if node is shadowed on left	*/
  	
  	int leftbound(struct node *pntr)
  	{
  	  int isleft;
  	  
  	  isleft = 0;
  	  if ((*pntr).last.x < -WIDTH/2)
  	  	isleft = 1;
  	  	
  	  return(isleft);
  	}
  
  	/* Determine if node is shadowed on right	*/
  	
  	int rightbound(struct node *pntr)
  	{
  	  int isright;
  	  
  	  isright = 0;
  	  if ((*pntr).last.x > WIDTH/2)
  	  	isright = 1;
  	  	
  	  return(isright;)
  	}
\end{lstlisting}